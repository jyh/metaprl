% This is "sig-alternate.tex" V1.3 OCTOBER 2002
% This file should be compiled with V1.6 of "sig-alternate.cls" OCTOBER 2002
%
% This example file demonstrates the use of the 'sig-alternate.cls'
% V1.6 LaTeX2e document class file. It is for those submitting
% articles to ACM Conference Proceedings WHO DO NOT WISH TO
% STRICTLY ADHERE TO THE SIGS (PUBS-BOARD-ENDORSED) STYLE.
% The 'sig-alternate.cls' file will produce a similar-looking,
% albeit, 'tighter' paper resulting in, invariably, fewer pages.
%
% ----------------------------------------------------------------------------------------------------------------
% This .tex file (and associated .cls V1.6) produces:
%       1) The Permission Statement
%       2) The Conference (location) Info information
%       3) The Copyright Line with ACM data
%       4) NO page numbers
%
% as against the acm_proc_article-sp.cls file which
% DOES NOT produce 1) thru' 3) above.
%
% Using 'sig-alternate.cls' you have control, however, from within
% the source .tex file, over both the CopyrightYear
% (defaulted to 2002) and the ACM Copyright Data
% (defaulted to X-XXXXX-XX-X/XX/XX).
% e.g.
% \CopyrightYear{2003} will cause 2002 to appear in the copyright line.
% \crdata{0-12345-67-8/90/12} will cause 0-12345-67-8/90/12 to appear in the copyright line.
%
% ---------------------------------------------------------------------------------------------------------------
% This .tex source is an example which *does* use
% the .bib file (from which the .bbl file % is produced).
% REMEMBER HOWEVER: After having produced the .bbl file,
% and prior to final submission, you *NEED* to 'insert'
% your .bbl file into your source .tex file so as to provide
% ONE 'self-contained' source file.
%
% ================= IF YOU HAVE QUESTIONS =======================
% Questions regarding the SIGS styles, SIGS policies and
% procedures, Conferences etc. should be sent to
% Adrienne Griscti (griscti@acm.org)
%
% Technical questions _only_ to
% Gerald Murray (murray@acm.org)
% ===============================================================
%
% For tracking purposes - this is V1.3 - OCTOBER 2002

% EMRE: use the preprint option to get rid of the silly ACM copyright.
\documentclass[preprint,letterpaper]{sig-alternate}

\usepackage[dvipdfm]{hyperref}
\usepackage{makeidx}
\makeindex
\usepackage{draftfooter}\pagestyle{draft}\newcommand{\thepage}{\arabic{page}}

%%%%%%%%%%%%%%%%%%%%%%%%%%%%%%%%%%%%%%%%%%%%%%%%%%%%%%%%%%%%%%%%%%%%%%%%%
% METAPRL DEFINITIONS
%%%%%%%%%%%%%%%%%%%%%%%%%%%%%%%%%%%%%%%%%%%%%%%%%%%%%%%%%%%%%%%%%%%%%%%%%

%
% Main tex flag is used for commenting MetaPRL regions.
%
\newif\iftex
\texfalse

\newcommand\texoff{\texfalse}

%
% Make sure \|,\< and \> always produce an appropriate symbol
%
\def\|{\ifmmode\mid\else$\mid$\fi}
\def\lt{\ifmmode<\else\texttt{<}\fi}
\def\gt{\ifmmode>\else\texttt{>}\fi}
\def\makehat{\ifmmode\hat{\ }\else\^{}\fi}

%
% Sectioning commands
%
\newcommand\theory[1]{\chapter{#1}}
\newcommand\labelmodule[2]{\section{#2 module}
  \hypertarget{hypmodule:#1}{}
  \label{module:#1}}
\newcommand\modsection[1]{\subsection{#1}}
\newcommand\modsubsection[1]{\subsubsection{#1}}
\newcommand\labelchapter[2]{\chapter{#2}
  \hypertarget{hypsection:#1}{}
  \label{section:#1}}
\newcommand\labelsection[2]{\section{#2}
  \hypertarget{hypsection:#1}{}
  \label{section:#1}}
\newcommand\labelsubsection[2]{\subsection{#2}
  \hypertarget{hypsection:#1}{}
  \label{section:#1}}
\newcommand\labelsubsubsection[2]{\subsubsection{#2}
  \hypertarget{hypsection:#1}{}
  \label{section:#1}}

\newcommand\hreflabelmodule[2]{\hyperlink{hypmodule:#1}{{\tt #2}}}
\newcommand\reflabelmodule[2]{\ref{module:#1}}

\newcommand\hreflabelchapter[2]{\hyperlink{hypsection:#1}{{\tt #2}}}
\newcommand\reflabelchapter[2]{\ref{section:#1}}

\newcommand\hreflabelsection[2]{\hyperlink{hypsection:#1}{{\tt #2}}}
\newcommand\reflabelsection[2]{\ref{section:#1}}

\newcommand\hreflabelsubsection[2]{\hyperlink{hypsection:#1}{{\tt #2}}}
\newcommand\reflabelsubsection[2]{\ref{section:#1}}

\newcommand\hreflabelsubsubsection[2]{\hyperlink{hypsection:#1}{{\tt #2}}}
\newcommand\reflabelsubsubsection[2]{\ref{section:#1}}

\newcommand\parents{\thysection{Parents}}
\newcommand\terms{\thysection{Terms}}
\newcommand\resources{\thysection{Resources}}
\newcommand\tactics{\thysection{Tactics}}
\newcommand\tacticals{\thysection{Tacticals}}
\newcommand\rules{\thysection{Rules}}
\newcommand\rewrites{\thysection{Rewrites}}
\newcommand\conversions{\thysection{Conversions}}
\newcommand\conversionals{\thysection{Conversionals}}
\newcommand\convs{\thysection{Conversions}}

%
% Hyperlinks
%
\newcommand\labeltarget[2]{%
  \label{target:#1}
  \hypertarget{hyptarget:#1}{{\tt #2}}\index{Targets!#1@{\it #2}}}

\newcommand\labelterm[2]{%
  \label{term:#1}
  \hypertarget{hypterm:#1}{{\tt #2}}\index{Terms!#1@{\it #2}}}
\newcommand\labelresource[2]{%
  \label{resource:#1}
  \hypertarget{hypresource:#1}{{\tt #2}}\index{Resources!#1@{\it #2}}}
\newcommand\labelrewrite[2]{%
  \label{rewrite:#1}
  \hypertarget{hyprewrite:#1}{{\tt #2}}\index{Rewrites!#1@{\it #2}}}
\newcommand\labeltactic[2]{%
  \label{tactic:#1}
  \hypertarget{hyptactic:#1}{{\tt #2}}\index{Tactics!#1@{\it #2}}}
\newcommand\labelconv[2]{%
  \label{conv:#1}
  \hypertarget{hypconv:#1}{{\tt #2}}\index{Conversions!#1@{\it #2}}}
\newcommand\labelrule[2]{%
  \label{rule:#1}
  \hypertarget{hyprule:#1}{{\tt #2}}\index{Rules!#1@{\it #2}}}

\newcommand\hreflabeltarget[2]{\hyperlink{hyptarget:#1}{{\tt #2}}}
\newcommand\hreflabelterm[2]{\hyperlink{hypterm:#1}{{\tt #2}}}
\newcommand\hreflabelresource[2]{\hyperlink{hypresource:#1}{{\tt #2}}}
\newcommand\hreflabelrewrite[2]{\hyperlink{hyprewrite:#1}{{\tt #2}}}
\newcommand\hreflabeltactic[2]{\hyperlink{hyptactic:#1}{{\tt #2}}}
\newcommand\hreflabelconv[2]{\hyperlink{hypconv:#1}{{\tt #2}}}
\newcommand\hreflabelrule[2]{\hyperlink{hyprule:#1}{{\tt #2}}}

\newcommand\reftarget[1]{\pageref{target:#1}}
\newcommand\refterm[1]{\pageref{term:#1}}
\newcommand\refresource[1]{\pageref{resource:#1}}
\newcommand\refrewrite[1]{\pageref{rewrite:#1}}
\newcommand\reftactic[1]{\pageref{tactic:#1}}
\newcommand\refconv[1]{\pageref{conv:#1}}
\newcommand\refrule[1]{\pageref{rule:#1}}

%%%%%%%%%%%%%%%%%%%%%%%%%%%%%%%%%%%%%%%%%%%%%%%%%%%%%%%%%%%%%%%%%%%%%%%%
% MACROS
%%%%%%%%%%%%%%%%%%%%%%%%%%%%%%%%%%%%%%%%%%%%%%%%%%%%%%%%%%%%%%%%%%%%%%%%

%
% Quote a name
%
\newcommand\ms[1]{\hbox{{\it #1}}}

%
% Names
%
\newcommand\Nuprl{\textsf{Nuprl}}
\newcommand\NuPRL{\Nuprl}
\newcommand\MetaPRL{\textsf{MetaPRL}}
\newcommand\OCaml{\textsf{OCaml}}
\newcommand\MartinLof{Martin--L\"{o}f}

%
% Terms
%
\newcommand\univ{\mathbb{U}}
\newcommand\Type{{\it Type}}
\newcommand\Unit{{\it Unit}}
\newcommand\Void{{\it Void}}
\newcommand\Atom{{\it Atom}}
\newcommand\Bool{{\it Bool}}
\newcommand\Top{{\it Top}}

%
% Simple terms
%
\newcommand\xsequent[3]{[{\it #1}]\ #2 \vdash #3}
\newcommand{\sq}[1]{\mbox{{\large\bf[}}{#1}\mbox{{\large\bf]}}}

%
% A proof step
%
\newcommand\xtac[4]{%
\begin{array}{||l}
\begin{array}{l}
  #3
\end{array}\\
\multicolumn{1}{||@{}l}{\hbox{---\tt BY #1}\ {#2}}\\
\begin{array}{l}
  #4
\end{array}
\end{array}}

%
% A rule definition
%
\newcommand\defrule[4]{%
\begin{array}{ll}
\multicolumn{2}{l}{\hbox{\bf rule #1}\ #2 =}\\
\qquad &
  \begin{array}{l}
    #3
  \end{array}\\
& \begin{array}{l}
  #4
  \end{array}
\end{array}}


\setlength{\commentwidth}{0.5\textwidth}
\addtolength{\commentwidth}{-0.5\columnsep}

\renewcommand{\bottomfraction}{0.2}
\renewcommand{\floatpagefraction}{0.9}

%\commentfalse

\begin{document}

% EMRE: What to put instead of the ACM copyright.
\toappearbox{Submitted to MER$\lambda$IN 2003, Uppsala, Sweden.}

%
% --- Author Metadata here ---
%\conferenceinfo{ICFP}{'03 Uppsala, Sweden}
%\CopyrightYear{2003} % Allows default copyright year (2000) to be over-ridden - IF NEED BE.
%\crdata{0-12345-67-8/90/01}  % Allows default copyright data (0-89791-88-6/97/05) to be over-ridden - IF NEED BE.
% --- End of Author Metadata ---

\title{Compiler Implementation in a Formal Logical Framework\titlenote{An
  extended version of this paper is available as Caltech technical report
  \texttt{caltechCSTR:2003.002} at
  \href{http://resolver.caltech.edu/caltechCSTR:2003.002}{\texttt{http://caltechcstr.library.caltech.edu/}}}
}

%
% You need the command \numberofauthors to handle the "boxing"
% and alignment of the authors under the title, and to add
% a section for authors number 4 through n.
%
% Up to the first three authors are aligned under the title;
% use the \alignauthor commands below to handle those names
% and affiliations. Add names, affiliations, addresses for
% additional authors as the argument to \additionalauthors;
% these will be set for you without further effort on your
% part as the last section in the body of your article BEFORE
% References or any Appendices.

\numberofauthors{1}
%
% You can go ahead and credit authors number 4+ here;
% their names will appear in a section called
% "Additional Authors" just before the Appendices
% (if there are any) or Bibliography (if there
% aren't)

% Put no more than the first THREE authors in the \author command
\author{
%
% The command \alignauthor (no curly braces needed) should
% precede each author name, affiliation/snail-mail address and
% e-mail address. Additionally, tag each line of
% affiliation/address with \affaddr, and tag the
%% e-mail address with \email.
\alignauthor Jason Hickey, Aleksey Nogin, Adam Granicz, and Brian Aydemir \thanks{This
  work was supported in part by the DoD Multidisciplinary
  University Research Initiative (MURI) program administered by the
  Office of Naval Research (ONR) under Grant N00014-01-1-0765, the
  Defense Advanced Research Projects Agency (DARPA), the United States
  Air Force, the Lee Center, and by NSF Grant CCR 0204193.}\\
       \affaddr{California Institute of Technology}\\
       \affaddr{1200 E. California Blvd.}\\
       \affaddr{Pasadena, CA 91125, USA}\\
       \email{\{jyh,nogin,granicz,emre\}@cs.caltech.edu}
}
\maketitle

\begin{abstract}
The task of designing and implementing a compiler can be a
difficult and error-prone process.  In this paper, we present a new
approach based on the use of higher-order abstract syntax and term
rewriting in a logical framework.  All program transformations, from
parsing to code generation, are cleanly isolated and specified as term
rewrites.  This has several advantages.  The correctness of the
compiler depends solely on a small set of rewrite rules that are
written in the language of formal mathematics.  In addition, the logical
framework guarantees the preservation of scoping, and it automates
many frequently-occurring tasks including substitution and rewriting
strategies.  As we show, compiler development in a logical framework
can be \emph{easier} than in a general-purpose language like ML, in
part because of automation, and also because the framework provides
extensive support for examination, validation, and debugging of the
compiler transformations.  The paper is organized around a case study,
using the \MetaPRL{} logical framework to compile an ML-like language to
Intel x86 assembly.  We also present a scoped formalization of x86
assembly in which all registers are immutable.

\end{abstract}

% A category with the (minimum) three required fields
%\category{H.4}{Information Systems Applications}{Miscellaneous}
%A category including the fourth, optional field follows...
%\category{D.2.8}{Software Engineering}{Metrics}[complexity measures, performance measures]
%\terms{Delphi theory}

% EMRE: We don't strictly need keywords for this conference.
% \keywords{formal compilers}

\sloppy

\input{experimental/compile/theory-body.tex}

\bibliographystyle{plain}
\bibliography{../inputs/rc}

\printindex

\balancecolumns
\end{document}
