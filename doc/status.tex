\documentclass{article}

\usepackage{proof}

\begin{document}

\section{Nuprl-Light Description}

Nuprl-Light has these features that distinguish it from other provers.

\begin{itemize}
\item
The theory structure is modular.  Different theories are defined as
separate modules.  Current theories include:
\begin{enumerate}
\item
ITT: Nuprl type theory,
\item
CZF: Aczel's constructive set theory,
\item
LF: initial implementation of Edinburgh Logical Framework.
\end{enumerate}

Theories can be {\em related\/} by showing how proofs from one
theory can be transformed to proofs in the other.  Theories are
defined by their collection of inference rules and computational
rewrite rules.  The relation is given by defining a model, and
deriving the rules.

For instance, CZF is derived from ITT.  The relation is given by
defining the type of sets as a W-type in ITT, and then showing that
the CZF rules are valid rules in ITT.

\item
Nuprl-Light has close ties to the programming language.  Nuprl-Light
builds in the OCaml language, and Nuprl-Light modules are defined on
top of the OCaml module system.  As a result, Nuprl-Light supports
arbitrary interleaving of formal statements and program code.

\item
Inheritance of formal properties and tactics.  Nuprl-Light is a
tactic-based theorem prover like Nuprl, but tactics are defined
relative to the module system.  The ITT theory defines a collection of
standard taactics that are specific to ITT.  These tactics are
available in CZF, but CZF also defines new tactics that are available
only in CZF.
\end{itemize}

\section{Long Term Goals}

\subsection{Proof Procedures}
Incorporation of proof procedures, especially thos that generate
proofs for their results.  These include:
\begin{description}
\item[Arithmetic]
SupInf, Loop Residue, Omega test.
\item[Model checkers]
Hardware checkers like espresso,
software checker like Holtzman's SPIN.
\item[Symbolic manipulation].
\end{description}

Currently there is a framework for incorporating arithemtic into ITT,
in the itt\_arith module.  This module provides a normalized
representation of a sequent.  This representation can be used to
implement procedures like SupInf.

\subsection{Proof automation}
The current tactics in Nuprl-Light are pretty simple (including D,
Eqcd, etc).  There is no Auto tactic.  With the module system we should
be able to provide sophisticated automation procedures that rely on
domain-specific knowledge.

\subsection{Domain-specific theories}
Some problems can use a simpler logic that the full ITT.  These
domain-specific logics allow different types of proof search.  For
example, some problems are expressible in first-order logic, which has
efficient proof procedures, and simplified well-formedness
restrictions.

\subsection{Relations between theories}
We already have a relation to CZF.  Other interesting relations are to
other systems, like HOL, PVS, Coq, and Isabelle.  Sophisticaated
relations like these would allow us to import results from these
systems into Nuprl-Light.

\subsection{Reflection}
We want to be able to express reflection in Nuprl-Light.  The module
system allows the theory to be reflected to be precisely defined, and
we can develop reflection incrementally starting with small
subtheories of ITT.

Among other things, reflection would allow us to give a formal account
of procedures like SupInf, and to make proof-theoretic arguments,
including arguments about computational complexity.

\subsection{Closer ties to ML}
The semantics of OCaml is given in theories/ocaml and
theories/ocaml\_sos.  These can be developed further to give a stronger
connection between the formal Nuprl-Light theories and OCaml,
including
\begin{itemize}
\item
operational semantics,
\item
Relation to ITT.
\end{itemize}

\section{Problems}

Here are some short-term problems.

\subsection{Arithmetic and decision procedures}
We don't have any tactics for arithmetic.  We have implemented a
SupInf algorithm in ML, and we have the framework for adding SupInf to
Nuprl-Light (in the itt\_arith module).  The next step is to provide a
version of SupInf that produces a primitive Nuprl-Light proof when it
is successful.

This is related to the general problem of decision procedures, and
eventually the problem of reflection.

\subsection{Proof automation}
In general we need more sophisticated tactics to automate proofs
search.  Specific problems are proof automation for (intutionistic)
propositional logic, first-order logic, which we may be able to extend
to higher-order logics.

The initial goal would be to have tactics that solve problems in
propositional logic, for example.  Eventually, we would incorporate
these into Nuprl-Light's forward chaining mechanism, which would allow
them to run in the background, watching for solvable goals.

\subsection{Derived Rules}
Right now, derivations are by backward-chaining only.  That is,
suppose we are deriving an elimination rule like the following.

$$\infer{\Gamma, A \Rightarrow B, \Delta \vdash T}{\Gamma A
  \Rightarrow B, \Delta \vdash A & \Gamma, A
  \Rightarrow B, \Delta, B \vdash T}$$

The method of proof allows for backward chaining from the goal
$\Gamma, A \Rightarrow B, \Delta \vdash T$, but it does not allow any
forward chaining from the assumptions.  Some larger examples are given
in the CZF theory.

The problem is figure out and implement a forward chaining mechanism.

\subsection{Theory redundancy}
Figure out howmuch of the ITT theory is redundant and derive proofs
for the redundant parts.  For example, and the rules for non-dependent
functions are derivable from dependent functions.

\end{document}

