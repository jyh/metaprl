\documentclass{sig-alternate}

\conferenceinfo{MERLIN '03}{August 26, 2003 Uppsala, Sweden}
\CopyrightYear{2003}
\global\acmcopyr={1-58113-800-8/03/0008}

\usepackage{makeidx}
\makeindex
%\usepackage{draftfooter}\pagestyle{draft}

\input{metaprl}
\usepackage{url}
\usepackage{nameref}
\usepackage{hyperref}

%\commentfalse

\begin{document}

\title{Compiler Implementation in a Formal Logical Framework\titlenote{An
  extended version of this paper is available as Caltech technical report
  \texttt{caltechCSTR:2003.002} at
  \href{http://resolver.caltech.edu/caltechCSTR:2003.002}{\texttt{http://caltechcstr.library.caltech.edu/}}}
}

\numberofauthors{1}

\author{
\alignauthor Jason Hickey, Aleksey Nogin, Adam Granicz, and Brian Aydemir
\titlenote{This
  work was supported in part by the DoD Multidisciplinary
  University Research Initiative (MURI) program administered by the
  Office of Naval Research (ONR) under Grant N00014-01-1-0765, the
  Defense Advanced Research Projects Agency (DARPA), the United States
  Air Force, the Lee Center, and by NSF Grant CCR
  0204193.}\\
\affaddr{California Institute of Technology, M/C 256-80}\\
\affaddr{1200 E. California Blvd.}\\
\affaddr{Pasadena, CA 91125, USA}\\
\email{\{jyh,nogin,granicz,emre\}@cs.caltech.edu}
}
\maketitle

\begin{abstract}
The task of designing and implementing a compiler can be a
difficult and error-prone process.  In this paper, we present a new
approach based on the use of higher-order abstract syntax and term
rewriting in a logical framework.  All program transformations, from
parsing to code generation, are cleanly isolated and specified as term
rewrites.  This has several advantages.  The correctness of the
compiler depends solely on a small set of rewrite rules that are
written in the language of formal mathematics.  In addition, the logical
framework guarantees the preservation of scoping, and it automates
many frequently-occurring tasks including substitution and rewriting
strategies.  As we show, compiler development in a logical framework
can be \emph{easier} than in a general-purpose language like ML, in
part because of automation, and also because the framework provides
extensive support for examination, validation, and debugging of the
compiler transformations.  The paper is organized around a case study,
using the \MetaPRL{} logical framework to compile an ML-like language to
Intel x86 assembly.  We also present a scoped formalization of x86
assembly in which all registers are immutable.

\end{abstract}

\sloppy
%\raggedbottom
\renewcommand{\floatpagefraction}{0.9}

\input{theory-body.tex}

\bibliographystyle{plain}
\bibliography{rc}

\printindex
\balancecolumns

\end{document}
